

\section{Programa en C}
Lo que primero que se ha realizado, antes de escribir ningún código, es pensar en qué se necesita para poder realizar las medidas que queremos:

\begin{enumerate}
    \item Programa a medir.
    \item Programa con el que medir.
\end{enumerate}

El primer punto está claro y los motivos por los cuales se han escogido ya se han explicado. Lo que falta, es elegir el segundo apartado: el programa con el cual medir.

Si bien, en la asignatura de \textit{Arquitectura e Ingeniería de Computadores}, se vieron algunos benchmarks con los cuales se puede medir el rendimiento de todo el sistema. Lo que se busca ahora es algo más específico: se busca poder medir el programa de una manera controlada. Con lo cual, benchmarks como el \textit{SPEC CPU 2017}, no es lo que se busca.

Por lo tanto, se pasa al siguiente nivel de detalle: medir los contadores hardware del sistema. Y, un programa/aplicación que se conoce y se ha usado es la librería PAPI (Performance Application Programming Interface).

---

Lo que se ha hecho es tomar algunas funciones de PAPI (aquellas que son necesarias) y crear un pequeño \textit{wrap} sobre ellas e incorporarlas a una librería compartida propia. Ésta tiene el nombre de TODO: y en ella se añadirán además las funciones que serán llamadas desde Python y que hacen uso de las funciones de bajo nivel de PAPI.

[Contar/Mostrar el ]


\section{Interfaz ctypes}

Como se ha comentado, PAPI es una librearía escrita en C/C++ y lo que se quiere es trabajar con ella en aplicaciones de Deep Learning con Python.
Si bien es cierto que, existen aplicaciones de Machine Learning para prácticamente todos los lenguajes de programación (C, C++, Java, Clojure, etc.), lo que se busca es sacar provecho a la abundancia de aplicaciones que existen en Python, y poder analizar el rendimiento/comportamiento del sistema durante su ejecución.

Para ello, se ha buscado una manera de poder portar PAPI a Python, y la forma/manera más sencilla es mediante el uso de la librería ctypes; que permite llamar funciones de librerías compartidas (o DLLs) y crear tipos de datos que son compatibles con el lenguaje C.

\subsection{Problemas}
El porte de 


\section{Medidas multithreading con Python}



\section{Keras callbacks}

- qué es un callbacks
tu callbacks

